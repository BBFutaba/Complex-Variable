\documentclass[12pt]{article}
\pagestyle{empty}
\usepackage{amsmath, amssymb, amsthm}
\usepackage{latexsym, epsfig, ulem, cancel, multicol, hyperref}
\usepackage{graphicx, tikz, subfigure,pgfplots}
\usepackage[margin=1in]{geometry}
\setlength{\parindent}{0pt}
\usepackage{multirow}
\usepackage{mathtools}


\newcommand{\R}{\mathbb{R}}
\newcommand{\dydx}{\frac{dy}{dx}}
\usepackage{verbatim}
\usepackage{tikz}
\usepackage{pgfplots}

\newcommand{\wsnumber}{1}
\newcommand{\wstopic}{Vectors}
\pgfplotsset{
    every linear axis/.append style={
       axis x line=center,
       axis y line=center,
       xlabel={$x$},
       ylabel={$y$}
    },
    every axis plot/.append style={thick,mark=none}
}
\tikzset{
    point/.style={circle,draw,fill,minimum width=0.3ex,inner sep=0pt,outer sep=0pt},
    every label/.append style={black}
}


\usepackage[margin=1in]{geometry}
\usepackage{amsmath, amssymb, amsthm, graphicx, hyperref}
\usepackage{enumerate}
\usepackage{fancyhdr}
\usepackage{multirow, multicol}
\usepackage{tikz}
\pagestyle{fancy}
\fancyhead[RO]{Spring 2024}
\fancyhead[LO]{MA-UY 3113 Complex Variables and Linear Algebra }
\usepackage{comment}
\newif\ifshow
\showfalse

\ifshow
  \newenvironment{solution}{\textbf{Solution.}}{}
\else
  \excludecomment{solution}
\fi

\renewcommand{\thefootnote}{\fnsymbol{footnote}}
\usepackage{comment}


\newtheorem*{remark}{Remark}

\newcommand*{\GridSize}{4}

\newcommand*{\ColorCells}[1]{% #1 = list of x/y/color
  \foreach \x/\y/\color in {#1} {
    \node [fill=\color, draw=none, thick, minimum size=1cm] 
      at (\x-.5,\GridSize+0.5-\y) {};
    }%
}%

\begin{document}

\begin{center}
\ifshow
  \textbf{\Large Homework 8 Solution}\\
\else
  \textbf{\Large Homework 8}\\
\fi
Due: Tuesday May 14, by 11:59pm,\\via Gradescope\\
\end{center}

\hrule

\vspace{0.2cm}

\begin{enumerate}[$\bullet$]
\item  {\textbf{\textit{Note that you must assign a page to each problem you submit.}}}   Gradescope has great YouTube videos available on how to submit homework.  \textit{\textbf{Failure to submit homework correctly will result in a zero on homework.}}
\item No LaTeX for this homework.  
\item Late homework is not accepted.  Lateness due to technical issues will not be excused.  
\end{enumerate}

\hrule

\vspace{0.5cm}



\begin{enumerate}


\item Let $T:V \rightarrow W$ be a linear mapping where $V$ and $W$ are vectors spaces.  Show that $T(0_{v}) = 0_{w}$.  Here $0_{v}, \; 0_{w}$ are the zero elements in $V, \; W$ respectively.  

\item Determine where the following transformations are linear/non-linear.  If the transformation is non-linear, then provide an appropriate counterexample.  
\begin{enumerate} 
 \item    $T\left(\begin{bmatrix} 
 x_1\\x_2\\x_3 \end{bmatrix} \right)=\begin{bmatrix} x_1-x_2\\x_2+x_3^2 \end{bmatrix} $.
   \item $T:\mathbb{R}^3\rightarrow\mathbb{R}^2$, where $T\left(\begin{bmatrix} x_1\\x_2\\x_3 \end{bmatrix} \right)=\begin{bmatrix} x_1+x_3\\-2x_2 \end{bmatrix} $.
 
 \item     $T\left(\begin{bmatrix} x_1\\x_2\\x_3 \end{bmatrix} \right)=\begin{bmatrix} x_1-x_2\\x_2+x_3\\1+x_3 \end{bmatrix} $.

   \item  $T:\mathbb{R}^3\rightarrow\mathbb{R}^4$, where $T\left(\begin{bmatrix} x_1\\x_2\\x_3 \end{bmatrix} \right)=\begin{bmatrix} 2x_1-x_3\\x_2+x_3\\0\\x_1-3x_2 \end{bmatrix} $.

\item     $T:\mathbb{P}_2\rightarrow\mathbb{P}_3$, where $T\left(p\left(x\right)\right)=xp\left(x\right)$.

    \item Let $T:\mathbb{R}^2\rightarrow\mathbb{R}^3$ where $T\left(\begin{bmatrix}x_1\\x_2 \end{bmatrix} \right)=\begin{bmatrix} x_1-x_2\\x_1\\2x_1+x_2 \end{bmatrix} $
\end{enumerate} 


\item Let  $\beta = \{ 2x^{2}, 1 + x, x^{2} + 2x - 3\}$ and $\beta' = \{ x, x^{2}, 2-x + 5x^{2}\}$ be ordered bases for the vector space $P_{2}(\mathbb{R})$.  Find the change of basis matrix from $\beta$ to $\beta'$

\item 

\begin{enumerate}
    \item Let $\mathcal{B}=\left\{\begin{bmatrix} 1\\0 \end{bmatrix},\begin{bmatrix} 0\\1 \end{bmatrix} \right\}$ and $\Tilde{\mathcal{B}}=\left\{\begin{bmatrix} 1\\1\\0\end{bmatrix} ,\begin{bmatrix} 0\\1\\1 \end{bmatrix} ,\begin{bmatrix} 2\\2\\3 \end{bmatrix} \right\}$ be the ordered basis for the domain and the co-domain respectively. Find the matrix representation of $T$, where $T$ is the linear operator in problem $10(f)$
    

\item     Change the ordered bases of the domain to $\beta=\left\{\begin{bmatrix} 1\\2 \end{bmatrix} ,\begin{bmatrix} 2\\3 \end{bmatrix} \right\}$, whilst keeping the ordered basis for the co-domain same as in part (a). What is the matrix associated with the transformation now?
\end{enumerate}  
 




\item  Let $T: P_{3}(\mathbb{R}) \rightarrow P_{3}(\mathbb{R})$ be the linear operator 
    \[
    Tp = (2x + 1)\frac{dp}{dx} + x^{2} \frac{d^{2}p}{dx^{2}} 
    \]
    
    
    $\beta = \{ 1, x, x^{2}, x^{3} \}$ be an ordered bases for the domain and co-domain of $T$. 
    \begin{enumerate}
        \item Find $[T]$ with respect to the given ordered basis.  
        \item Find the eigenvalues of $T$ by finding the eigenvalues of $[T]$.
        \item Find the corresponding eigenvectors of $T$ by finding the corresponding eigenvectors of $[T]$.  
    \end{enumerate}  

    
    
    \item    Let $T:V \rightarrow W$ be a linear mapping where $V$ and $W$ are vectors spaces.
    Let $\beta = \{v_{1}, v_{2}, v_{3}\}$ be an ordered bases of the domain of $T$ and $\beta' = \{ w_{1}, w_{2}, w_{3} \}$ be an ordered bases for the co-domain of $T$.  Assume 
    \[
    Tv_{1} = w_{2}, \; Tv_{2} = Tv_{3} = w_{1} + w_{3} 
    \]
    Find 
    \[
    [T] \begin{bmatrix} 1 \\ 1 \\ 1 \end{bmatrix} 
    \]
    where [T] is the matrix representation of $T$ with respect to the ordered bases $\beta, \; \beta'$. 







\item Let $\omega = \exp\left( - \frac{2\pi i}{N} \right)$.  Then $\omega^{N} = 1$, i.e. $\omega$ is an $N$th root of unity.  
\begin{enumerate} 
\item Show that 
\[
\sum_{m=0}^{N-1}\omega^{m} = 0
\]
\item Let $w = \omega^{k}$ where $k$ is any posisitve integer such that $k < N -1$.  Show that 
\[
\sum_{m=0}^{N-1} w^{m} = 0
\]
\item The discrete Fourier transform matrix is a unitary matrix.   Find $\hat{x}$ where $x = \begin{bmatrix} 1 \\ \bar{\omega} \\ \bar{\omega}^{2} \\ \bar{\omega}^{3} \end{bmatrix}$ and $\omega = \exp( - i\pi/2)$.  Could you have guessed the answer?  Explain.  
\item Find the eigenvalues of the discrete Fourier transform matrix $F_{4}$.  I would encourage you to think about how we got the eigenvalues of the shift matrix homework.   

	
\end{enumerate}


\item Let 
\[
A  =  \begin{bmatrix} 6 & 18 & 3   \\ 2 & 12 & 1  \\ 4 & 15 & 3   \end{bmatrix} 
\]
\begin{enumerate}
    \item Find the matrices $L, \; U$ in the $LU-$decomposition of $A$.
    \item Let's now use the $LU-$decomposition.  You are asked to find $\vec{x}$ that satisfies equation  $A \vec{x} = \vec{b}$ where 
    \[
    \vec{b} =  \begin{bmatrix} 3 \\ 19 \\ 0    \end{bmatrix} 
    \]
    First solve the equation $L \vec{z} = \vec{b}$.  The follow that by solving the equation $U\vec{w} = \vec{z}$.  Then $\vec{x} = \vec{w}$.  
    \begin{remark}  
    Notice that $U\vec{w} = \vec{z}$ implies $LU \vec{w} = L \vec{z} = \vec{b}$.  Since $A = LU$ it must follow that $\vec{w} = \vec{x}$.  
    \end{remark} 
\end{enumerate}


\end{enumerate}

\end{document}