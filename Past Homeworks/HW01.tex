\documentclass[12pt]{article}
\pagestyle{empty}
\usepackage{amsmath, amssymb, amsthm}
\usepackage{latexsym, epsfig, ulem, cancel, multicol, hyperref}
\usepackage{graphicx, tikz, subfigure,pgfplots}
\usepackage[margin=1in]{geometry}
\setlength{\parindent}{0pt}
\usepackage{multirow}
\usepackage{mathtools}


\newcommand{\R}{\mathbb{R}}
\newcommand{\dydx}{\frac{dy}{dx}}
\usepackage{verbatim}
\usepackage{tikz}
\usepackage{pgfplots}

\newcommand{\wsnumber}{1}
\newcommand{\wstopic}{Vectors}
\pgfplotsset{
    every linear axis/.append style={
       axis x line=center,
       axis y line=center,
       xlabel={$x$},
       ylabel={$y$}
    },
    every axis plot/.append style={thick,mark=none}
}
\tikzset{
    point/.style={circle,draw,fill,minimum width=0.3ex,inner sep=0pt,outer sep=0pt},
    every label/.append style={black}
}


\usepackage[margin=1in]{geometry}
\usepackage{amsmath, amssymb, amsthm, graphicx, hyperref}
\usepackage{enumerate}
\usepackage{fancyhdr}
\usepackage{multirow, multicol}
\usepackage{tikz}
\pagestyle{fancy}
\fancyhead[RO]{Dennis Li}
\fancyhead[LO]{MA-UY 3113 Complex Variables and Linear Algebra }
\usepackage{comment}
\newif\ifshow
\showfalse

\ifshow
  \newenvironment{solution}{\textbf{Solution.}}{}
\else
  \excludecomment{solution}
\fi

\renewcommand{\thefootnote}{\fnsymbol{footnote}}
\usepackage{comment}


\newtheorem*{remark}{Remark}


\begin{document}

\begin{center}
\ifshow
  \textbf{\Large Homework 1 Solution}\\
\else
  \textbf{\Large Homework 1}\\
\fi
Due: Friday February 2, by 11:59pm,\\via Gradescope\\
\end{center}

\hrule

\vspace{0.2cm}

\begin{enumerate}[$\bullet$]
\item  {\textbf{\textit{Note that you must assign a page to each problem you submit.}}}   Gradescope has great YouTube videos available on how to submit homework.  \textit{\textbf{Failure to submit homework correctly will result in a zero on homework.}}
\item Problems that appear with the notation \colorbox{yellow}{$\ast$} will require you to TeX your solution.  If no highlighted star appears, then a hand written solution is OK.  
\item Late homework is not accepted.  Lateness due to technical issues will not be excused.  
\end{enumerate}

\hrule

\vspace{0.5cm}



\begin{enumerate}

\item (5 points) \colorbox{yellow}{$\ast$} Let $z_{1} = 3 -4i$ and $z_{2} = 1 + 2i$.  Find the complex number $z_{1}/z_{2}$ by setting $z_{1}/z_{2} = x + iy$ and solve for $x$ and $y$.  Do not use the conjugate at any point.  
\textbf{Solution:}\\
let


$\frac{3-4i}{1+2i}=x+iy$

we have:\\

\[
3-4i=(x+iy)(1+2i)\]\\
expand, we have:\\

\[3-4i=x+2ix+iy-2y\]
let \[Re(z_1/z_2)=x-2y\] \\
and \[Im(z_1/z_2)=2x+y\]
We have:\\
\[x-2y=3\]
\[2x+y=-4\]
solve for $x$,$y$ we have
\[x=-1\]
\[y=-2\]
so
\[
\frac{z_1}{z_2}=-1-2i
\]
\item (5 points)  Let $z_{1} = 3 -4i$ and $z_{2} = 1 + 2i$.  Find the complex number $z_{1}/z_{2}$ by using the conjugate of $z_{2}$.
\textbf{Solution:}
let\\\[
z_3=\frac{z_1}{z_2}=\frac{3-4i}{1+2i}
\]
we have the conjugate\\
\[\Bar{z_2}=1-2i\]
we multiply $z_3$ by $\Bar{z_2}$ and obtain:
\[
z_3=\frac{(3-4i)(1-2i)}{1+4}
\]
simplify:
\[
\frac{-5-10i}{5}=-1-2i
\]
and we have:
\[z_3=-1-2i\]

\item (5 points)  \colorbox{yellow}{$\ast$} Find all possible values of $\sqrt{-2+3i}$ without using polar coordinates.  That is, set $\sqrt{-2+3i} = x+iy$ and solve for $x$ and $y$.  
\\\textbf{Solution:}
let
\[
\sqrt{-2+3i}=x+iy
\]
square both sides
\[
-2+3i=x^2+2ixy-y^2
\]
similarly to question 1, we can obtain:
\[
-2=x^2-y^2
\]
\[
3=2xy
\]
square both sides again, we can obtain
\[
4=x^4-2x^2y^2+y^4\]
\[
9=4x^2y^2
\]
add the second equation to the first to obtain
\[
13=x^4+2x^2y^2+y^2
\]
which means
\[
\sqrt{13}=x^2+y^2 ... (1)
\]
\[
-2=x^2-y^2 ... (2)
\]
add the second equation to the first, we have
\[
-2+\sqrt{13}=2x^2
\]
\[
x=\pm \sqrt{\frac{-2+\sqrt{13}}{2}}
\]
substitute the solutions of $x$ back to the original equation (1), we can obtain
\[
-2=\frac{-2+\sqrt{13}}{2}-y^2
\]
\[
-2-\frac{-2+\sqrt{13}}{2}=-y^2
\]
\[
y=\pm \sqrt{\frac{2+\sqrt{13}}{2}}
\]
the hyperbolas intersects at the first and third quadrant, so we can determine the final solution to be:
\[
x_1=\sqrt{\frac{-2+\sqrt{13}}{2}}
\]
\[
y_1= \sqrt{\frac{2+\sqrt{13}}{2}}
\]
and
\[
x_2=-\sqrt{\frac{-2+\sqrt{13}}{2}}
\]
\[
y_2= -\sqrt{\frac{2+\sqrt{13}}{2}}
\]
so
\[z_1=x_1+iy_1
\]\[
z_2=x_2+iy_2
\]
satisfy $\sqrt{-2+3i}$

\item (5 points) Find all possible values of $\sqrt{-2+3i}$ using polar coordinates.\\
\textbf{Solution:}\\
let $z_1=-2+3i$
\[
|z_1|=\sqrt{4+9}=\sqrt{13}
\]
\[
\phi=arctan(\frac{3}{-2})+\pi
\]
$z_1$ can be written as:\\
\[
z_1=\sqrt{13}\cdot\exp{(arctan(\frac{3}{-2})+\pi)}
\]
let $z_2=Re^{i\theta}$, such that
\[
z_2=\sqrt{z_1}
\]
we can obtain
\[
(Re^{i\theta})^2=\sqrt{13}\cdot\exp{(arctan(\frac{3}{-2})+\pi)}
\]
\[
R^2e^{2i\theta}=\sqrt{13}\cdot\exp{(arctan(\frac{3}{-2})+\pi)}
\]
we can determine that
\[
R^2=\sqrt{13}
\]
\[
R=13^{\frac{1}{4}}
\]
discard the negative root since distance cannot be negative\\\\
furthermore
\[
\because 2\theta-\phi=2\pi k, k\in\mathbb{Z} 
\]
\[
\therefore \theta=\frac{arctan(\frac{3}{-2})+\pi}{2}+\pi k,k\in\mathbb{Z}
\]
we can conclude that, all possible results of $\sqrt{z_1}$ is
\[
z_2=13^{\frac{1}{4}}\cdot \exp(i[\frac{arctan(\frac{3}{-2})+\pi}{2}+\pi k]), k\in\{0,1\}
\]
\end{enumerate}




\end{document}