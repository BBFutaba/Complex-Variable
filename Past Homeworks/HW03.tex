\documentclass[12pt]{article}
\pagestyle{empty}
\usepackage{amsmath, amssymb, amsthm}
\usepackage{latexsym, epsfig, ulem, cancel, multicol, hyperref}
\usepackage{graphicx, tikz, subfigure,pgfplots}
\usepackage[margin=1in]{geometry}
\setlength{\parindent}{0pt}
\usepackage{multirow}
\usepackage{mathtools}


\newcommand{\R}{\mathbb{R}}
\newcommand{\dydx}{\frac{dy}{dx}}
\usepackage{verbatim}
\usepackage{tikz}
\usepackage{pgfplots}

\newcommand{\wsnumber}{1}
\newcommand{\wstopic}{Vectors}
\pgfplotsset{
    every linear axis/.append style={
       axis x line=center,
       axis y line=center,
       xlabel={$x$},
       ylabel={$y$}
    },
    every axis plot/.append style={thick,mark=none}
}
\tikzset{
    point/.style={circle,draw,fill,minimum width=0.3ex,inner sep=0pt,outer sep=0pt},
    every label/.append style={black}
}


\usepackage[margin=1in]{geometry}
\usepackage{amsmath, amssymb, amsthm, graphicx, hyperref}
\usepackage{enumerate}
\usepackage{fancyhdr}
\usepackage{multirow, multicol}
\usepackage{tikz}
\pagestyle{fancy}
\fancyhead[RO]{Dennis Li: zl2951}
\fancyhead[LO]{MA-UY 3113 Complex Variables and Linear Algebra }
\usepackage{comment}
\newif\ifshow
\showfalse

\ifshow
  \newenvironment{solution}{\textbf{Solution.}}{}
\else
  \excludecomment{solution}
\fi

\renewcommand{\thefootnote}{\fnsymbol{footnote}}
\usepackage{comment}


\newtheorem*{remark}{Remark}


\begin{document}

\begin{center}
\ifshow
  \textbf{\Large Homework 3 Solution}\\
\else
  \textbf{\Large Homework 3}\\
\fi
Due: Saturday March 2, by 11:59pm,\\via Gradescope\\
\end{center}

\hrule

\vspace{0.2cm}

\begin{enumerate}[$\bullet$]
\item  {\textbf{\textit{Note that you must assign a page to each problem you submit.}}}   Gradescope has great YouTube videos available on how to submit homework.  \textit{\textbf{Failure to submit homework correctly will result in a zero on homework.}}
\item Problems that appear with the notation \colorbox{yellow}{$\ast$} will require you to TeX your solution.  If no highlighted star appears, then a hand written solution is OK.  
\item Late homework is not accepted.  Lateness due to technical issues will not be excused.  
\end{enumerate}

\hrule

\vspace{0.5cm}



\begin{enumerate}


\item (5 points)  Show that 
\[
\lim_{z \rightarrow -1} Arg(z)  
\]
does not exists.  

\begin{remark}
You will have to construct two paths in the complex plane where $z \rightarrow -1$ along those paths and the limit value differs along those paths.  Note that $z=-1$ is not special; $f(z) = Arg(z)$ is not continous on the non-positive real axis, i.e. $(-\infty, 0]$.  
\end{remark}

\textbf{Solution:}\\
To show that the limit does not exist, we examine the limit along two different paths as \(z\) approaches \(-1\):

1. Along the real axis: \(z(t) = -1 + t\), with \(t\) real and \(t \to 0^+\).
   \[ \lim_{t \to 0^+} Arg(-1 + t) = \lim_{t \to 0^+} arctan(\frac{0}{-1+t}) = 0 \]

2. Along the line \(z(t) = -1 + it\), with \(t\) real and \(t \to 0^+\).
   \[ \lim_{t \to 0^+} Arg(-1 + it) = \lim_{t \to 0^+} arctan(\frac{t}{-1})+\pi=\pi \]

we can see that the two limits are not equal, such that \[\lim_{t \to -1} Arg(z)\] does not exist


\item (5 points) \colorbox{yellow}{$\ast$} Compute
\[
\lim_{z \rightarrow i}\frac{ z^{4} - 1}{z^{2} + 1} 
\]
\begin{remark}
FYI, the limit exists and I am asking you to find its value.   You do not have to simplify your final answer.
\end{remark}
\textbf{Solution:}\\
\[
\lim_{z\to i} \frac{ z^{4} - 1}{z^{2} + 1} = \lim_{z\to i} \frac{(z^2+1)(z^2-1)}{z^2+1}
\]
\[
=\lim_{z\to i} z^2-1=i^2-1=-2
\]


\item (10 points) 
\begin{enumerate}
    \item \colorbox{yellow}{$\ast$} Let $f(z) = \frac{1}{z}$.  Show that $f'(z) = \frac{-1}{z^{2}}$ using the definition of the derivative.  
\begin{remark}
    The definition of derivatives is\[
    f'(z)=\lim_{h\to 0} \frac{f(z+h)-f(z)}{h}
    \]
    where $ {h,z \in \mathbb{C}} $
    let $f(z)=\frac{1}{z}$, we have
    \[
    f'(z)=\lim_{h\to 0}\frac{\frac{1}{z+h}-\frac{1}{z}}{h}=\frac{\frac{z-(z+h)}{z(z+h)}}{h}
    \]
    \[
    =\lim_{h\to 0}\frac{-h}{h(z^2+zh)}=\lim_{h\to 0}\frac{-1}{z^2+zh}=\frac{-1}{z^2}
    \]
    
\end{remark}

\item  Let $f(z) = \frac{\bar{z}^{2}}{z}$ when $z \neq 0$ and $f(0) = 0$.  Use the definition of the derivative to show that $f$ is not differentiable at the origin. 
\end{enumerate}

\textbf{Solution:}
\\let $\Bar{z}=x-iy$, we have\[
f'(z)=\lim_{h\to 0} \frac{f(h)-f(0)}{h}
\]
we know $f(0) = 0$, hence this can be simplified as \[
\lim_{h\to 0} \frac{\Bar{h}^2}{h^2} = \lim_{h\to 0} (\frac{\Bar{h}}{h})^2
\]
evaluate $\lim_{h\to 0} \frac{\Bar{h}}{h}$,\[
\text{let $h=re^{i\theta}$}
\]
\[
\text{approach this limit from its exponential form}
\]\[
\lim_{r\to 0} (\frac{re^{-i\theta}}{re^{i\theta}})^2 = e^{-4i\theta}
\]
we can see how this limit can be a wide range of complex number by varying the parameter $\theta$, hence the limit does not exist\\
therefore this derivative does not exist


\item (10 points) 
\begin{enumerate}
    \item \colorbox{yellow}{$\ast$} Verify $f(z) = 1/z^{2}$ is differentiable for all $z \neq 0$ by verifying that the Cauchy-Riemann equations hold for all $z \neq 0$.  
\begin{remark}
    Cauchy-Riemann Equation:\\
    let $f(z)=u(z)+iv(z)$, and\[
    u_x(z_0)=v_y(z_0)
    \]\[
    u_y(z_0)=-v_x(z_0)
    \]then\[
    f'(z_0)=u_x(z_o)+iv_x(z_0)
    \]
\end{remark}\\
\textbf{Solution}
\\
\[
f(z)=f(x,y)=\frac{1}{(x+iy)^2}
\]
\[
\frac{1}{(x+iy)^2}=\frac{1}{x^2+2ixy-y^2}=-\frac{x^2 - y^2}{(x^2 + y^2)^2} + i\frac{2xy}{(x^2 + y^2)^2}
\]
\[
\frac{\partial u}{\partial x} = \frac{2x(x^2 + y^2)^2 - 2x(x^2 - y^2)2(x^2 + y^2)}{(x^2 + y^2)^4}
\]\[
\frac{\partial v}{\partial y} = \frac{-2(x^2 + y^2)^2 + 4xy^2(x^2 + y^2)}{(x^2 + y^2)^4}
\]\[
\frac{\partial u}{\partial y} = \frac{-2y(x^2 + y^2)^2 - 2(x^2 - y^2)2xy(x^2 + y^2)}{(x^2 + y^2)^4}
\]\[
\frac{\partial v}{\partial x} = \frac{-2(x^2 + y^2)^2 + 4x^2y(x^2 + y^2)}{(x^2 + y^2)^4}
\]
we can see that\[
u_x=v_y
\]
\[
u_y=-v_x
\]
and we can see that\[
\exists u_x, u_y, v_x, v_y \iff x,y\neq 0
\]
so\[
\exists f'(x,y) \iff x,y \neq 0
\]
    
    \item Verify $f(z) = z^{3}$ is differentiable for all $z$ by verifying that the Cauchy-Riemann equations hold for all $z$. 
\end{enumerate}
\textbf{Solution}
\[
f(x,y)=(x+iy)^3=x^3 - 3xy^2 + i(3x^2y - y^3)
\]
\[
u(x,y)=x^3-3xy^2
\]
\[
v(x,y)=3x^2y-y^3
\]
\[
u_x=3x^2-3y^2
\]
\[
u_y=-6xy
\]
\[
v_x=6xy
\]
\[
v_y=3x^2-3y^2
\]
we can see that\[
u_x=v_y
\]
\[
u_y=-v_x
\]
is true $\forall x,y\in \mathbb{R}$


\item (5 points) For complex number $z$, we define 
\[
\cos z = \frac{ e^{i z } + e^{-i z} }{2}, \; \; \sin z = \frac{ e^{i z } - e^{-i z } }{2i} 
\]
Show that $f(z) = \cos z$ and $ g(z) = \sin z$ are analytic in the complex plane by verifying that the Cauchy-Riemann equations hold for all $z$.  

\textbf{Solution}
\\
\[
\cos z = \frac{1}{2}e^{iz}+\frac{1}{2}e^{-iz}=\frac{1}{2}e^{i(x+iy)}+\frac{1}{2}e^{-i(x+iy)}
\]
by Euler's identity, we have
\[
cos z=\frac{1}{2}e^{ix}e^{-y}+\frac{1}{2}e^{-ix}e^{y}=\frac{1}{2}(e^{-y}\cos(x)+ie^{-y}\sin(x)+e^{y}\cos(-x)+ie^{y}\sin(-x))
\]
\[
u(x,y)=\frac{1}{2}(e^{-y}\cos(x)+e^{y}\cos(x))
\]
\[
v(x,y)=\frac{1}{2}(e^{-y}\sin(x)-e^{y}\sin(x))
\]
\[
u_x=\frac{1}{2}(-e^{-y}\sin(x)-e^{y}\sin(x)
\]
\[
u_y=\frac{1}{2}(-e^{-y}\cos(x)+e^{y}\cos(x))
\]
\[
v_x=\frac{1}{2}(e^{-y}\cos(x)-e^{y}\cos(x))
\]
\[
v_y=\frac{1}{2}(-e^{-y}\sin(x)-e^{y}\sin(x))
\]
we can see that\[
u_x=v_y
\]\[
u_v=-v_x
\]
is true $\forall x,y\in \mathbb{R}$


\item (5 points) Show that $f(z)$ is analytic in a region $D$ if and only if $\bar{f}(\bar{z})$ is analytic in a region $D$.  

\begin{remark}
   Try setting $f(z)  = u(x, y) + i v(x, y)$ and $\bar{f}(\bar{z})  = u(x, -y) -i v(x, -y)$ and work from there.   
\end{remark}

\textbf{Solution}
\\
Pf:\[
\text{if} 
\]
\[
\text{let $f(z)=u(x,y)+i(x,y)$} 
\]
\[
\Bar{f(z)}=u(x,-y)-iv(x,-y)
\]
$f(z)$ is analytic in D if $f(z)$ satisfy the Cauchy-Riemann Equation\[
u_x=v_y
\]\[
u_y=-v_x
\]for the given $f(z)=f(x,y)$, the following relationship has to be true
\[
\frac{\partial u}{\partial x}=\frac{\partial v}{\partial y}
\]\[
\frac{\partial u}{\partial y}=-\frac{\partial v}{\partial x}
\]now examine this relationship for $\Bar{f}(\Bar{z})$\\
\[
\frac{\partial u}{\partial x}=\frac{\partial (-v)}{\partial(-y)}
\]\[
\frac{\partial u}{\partial x}=\frac{\partial v}{\partial y}
\]
\[
\frac{\partial u}{\partial(-y)}=-\frac{\partial (-v)}{\partial (x)} \]
simplifying, we have\[
\frac{\partial u}{\partial y}=-\frac{\partial v}{\partial x}
\]
we can see that both derivations arrived at the same equality that is the Cauchy-Riemann equation\\
hence, if $f$ is analytic and satisfies this equation, $\Bar{f}(\Bar{z})$ also has to satisfy this equation\\
\[
\therefore \text{$f(z)$ is analytic in D} \iff \text{$\Bar{f}(\Bar{z})$ is analytic in D}
\]

\item (6 points)  Let $f(z) = u(z) + i v(z)$ be analytic in a region $D$ of the complex plane.  The function $v(z)$ is called the harmonic conjugate of $u$ in region $D$.  Harmonic conjugates are not unique.  For instance $v(z) + c$ where $c$ is any complex constant is also a harmonic conjugate of $u$.  Find a harmonic conjugate of   
\begin{enumerate}
    \item $u(z) = x^{2} - y^{2}$.  That is, find a function $v(z)$ such that $f(z) = x^{2} - y^{2} + i v(z)$ is analytic in the complex plane.  
    \\
    \textbf{Solution}\\
    $v(x,y)$ is said to be the harmonic conjugate of $u(x,y)$ if they are part of a holomorphic function\\
    that is, their relationship satisfy the Cauchy-Riemann equation.\\
    Therefore\[
    u_x=v_y
    \]\[
    u_y=-v_x
    \]let $u(x,y)=x^2-y^2$, we have the following relation\[
    2x=v_y(x,y)
    \]\[
    2y=v_x(x,y)
    \]we can integrating both side and obtain\[
    \int v_y(x,y)dy=v(x,y)+g(y)= 2xy+g(y)
    \]\[
    \int v_x(x,y)dx=v(x,y)+h(x)= 2xy+h(x)
    \]
    we can see that\[
    g(y)=h(x)
    \]which implies that this can only be a constant\[
    g(y)=h(x)=C, C\in\mathbb{R}
    \]and the harmonic conjugate of $u(x,y)$ is\[
    v(x,y)=2xy+C,C\in\mathbb{R}
    \]and is part of a holomorphic function\[
    f(z)=(x^2-y^2)+i(2xy+C), C\in\mathbb{R}
    \]

    
    \item  $u(z) = e^{x} \cos y$.  That is, find a function $v(z)$ such that $f(z) = e^{x} \cos y + i v(z)$ is analytic in the complex plane.
\\
\textbf{Solution}\\
similar to previous question, we can establish the following equation based on the Cauchy-Riemann Equation\[
u_x(x,y)=e^x\cos(y)=v_y(x,y)
\]\[
-u_v(x,y)=e^x\sin(y)=v_x(x,y)
\]we integrate both side as below\[
\int v_y(x,y)dy=e^x\sin(y)+g(x)
\]\[
\int v_x(x,y)dx=e^x\sin(y)+h(y)
\]we can see that\[
g(x)=h(y)=C, C\in\mathbb{R}
\]and thus we can determine that the harmonic conjugate of $u(x,y)$ is\[
v(x,y)=e^x\sin(y)+C,C\in\mathbb{R}
\]and is part of the holographic function\[
f(z)=e^x\cos(y)+i(e^x\sin(y)+C),C\in\mathbb{R}
\]

    
\end{enumerate}  

\begin{remark}
    Don't just state the function $v(z)$.  I want to see the work on how you found the harmonic conjugate.  Think Cauchy-Riemann equations.  
\end{remark}

\item \colorbox{yellow}{$\ast$} (9 points) Find 
\begin{enumerate}
    \item $\log(e^{z})$.  
    \begin{remark}
        Use the fact that $w = \log z$ if and only if $z = e^{w}$ and work from there.  Note that the answer is not $z$.   
    \end{remark}
    \textbf{Solution}\\
    \[
    \text{let $\log(e^z)=\alpha$, we raise both side to the power of $e$}
    \]\[
    e^z=e^\alpha
    \]and we know that
    \[
    e^{z+2i\pi k}=e^\alpha, k\in\mathbb{Z}
    \]
    \[
    \therefore \log(e^z)=\alpha+2i\pi k,k\in\mathbb{Z}
    \]
    \item $\log \; 1$
    \[
    \log(1)=z
    \]we can use the formula
    \[
    \log(z)=\log|z|+iarg(z)
    \]\[
    \therefore \log(1)=\log(1)+iarg(1), arg(1)=2\pi k, k\in \mathbb{R}
    \]
    \item Log  1.
    \\
    since the principle logarithm uses the principle argument of the complex number,\[
    -\pi \leq Arg(z) \leq \pi
    \]we can use the answer from the previous question except\[
    k=0
    \]
    \[
    \therefore Log(1)=0
    \]
\end{enumerate} 

\item (6 points) Let $z = 1-2i, \; w = -3 + 4i$.  Find 
\begin{enumerate}
    \item $z^{w}$
    \[
    z^w=(1-2i)^{(-3+4i)} \]we know that\[
    e^{w\log z}=z^w
    \]
    \[
    w\log z=(-3+4i)\log(1-2i)
    \]evaluate$\log(1-2i)$\[
    \log(1-2i)=\log(\sqrt{5})+i(\arctan(-2)+2\pi k),k\in\mathbb{Z}
    \]for simplicity's sake, we will agree that all $k$ in the following answer satisfy $k\in\mathbb{Z}$
    \\now include the complex coefficient\[
    w\log z=(-3+4i)(Log(\sqrt{5})+i(\arctan(-2)+2\pi k),k\in\mathbb{Z}
    \]\[
    =-3\log (\sqrt{5})-3i(\arctan (-2)+2\pi k)+i4\log(\sqrt{5})-4(\arctan (-2)+2\pi k))
    \]\[
    =-(3\log(\sqrt{5}+4(\arctan (-2)+2\pi k))+i(4\log(\sqrt{5})-3(\arctan (-2)+2\pi k))
    \]\[
    \therefore z^w = \exp{(-(3\log(\sqrt{5}+4(\arctan (-2)+2\pi k))+i(4\log(\sqrt{5})-3(\arctan (-2)+2\pi k))}
    \]
    \item $\mathcal{R}e\left(P.V. z^{w} \right)$ and $\mathcal{I}m\left(P.V. z^{w} \right)$
    from the answer above, we can see that:
    \[
    Re(z^w)=e^{-(3\log(\sqrt{5})+4(\arctan (-2)+2\pi k))}\cos(4\log(\sqrt{5})-3(\arctan (-2)+2\pi k))
    \]\[
    Im(z^w)=e^{-(3\log(\sqrt{5})+4(\arctan (-2)+2\pi k))}\sin(4\log(\sqrt{5})-3(\arctan (-2)+2\pi k))
    \]
\end{enumerate}
\begin{remark}
     Recall that P.V stands for principle value.  In part (b), I am asking for the real and imaginary parts of the principle value.  
\end{remark}





\end{enumerate} 

\end{document}