j\documentclass[12pt]{article}
\pagestyle{empty}
\usepackage{amsmath, amssymb, amsthm}
\usepackage{latexsym, epsfig, ulem, cancel, multicol, hyperref}
\usepackage{graphicx, tikz, subfigure,pgfplots}
\usepackage[margin=1in]{geometry}
\setlength{\parindent}{0pt}
\usepackage{multirow}
\usepackage{mathtools}


\newcommand{\R}{\mathbb{R}}
\newcommand{\dydx}{\frac{dy}{dx}}
\usepackage{verbatim}
\usepackage{tikz}
\usepackage{pgfplots}

\newcommand{\wsnumber}{1}
\newcommand{\wstopic}{Vectors}
\pgfplotsset{
    every linear axis/.append style={
       axis x line=center,
       axis y line=center,
       xlabel={$x$},
       ylabel={$y$}
    },
    every axis plot/.append style={thick,mark=none}
}
\tikzset{
    point/.style={circle,draw,fill,minimum width=0.3ex,inner sep=0pt,outer sep=0pt},
    every label/.append style={black}
}


\usepackage[margin=1in]{geometry}
\usepackage{amsmath, amssymb, amsthm, graphicx, hyperref}
\usepackage{enumerate}
\usepackage{fancyhdr}
\usepackage{multirow, multicol}
\usepackage{tikz}
\pagestyle{fancy}
\fancyhead[RO]{Dennis Li}
\fancyhead[LO]{MA-UY 3113 Complex Variables and Linear Algebra }
\usepackage{comment}
\newif\ifshow
\showfalse

\ifshow
  \newenvironment{solution}{\textbf{Solution.}}{}
\else
  \excludecomment{solution}
\fi

\renewcommand{\thefootnote}{\fnsymbol{footnote}}
\usepackage{comment}


\newtheorem*{remark}{Remark}


\begin{document}

\begin{center}
\ifshow
  \textbf{\Large Homework 6 Solution}\\
\else
  \textbf{\Large Homework 6}\\
\fi
Due: Sunday April 14, by 11:59pm,\\via Gradescope\\
\end{center}

\hrule

\vspace{0.2cm}

\begin{enumerate}[$\bullet$]
\item  {\textbf{\textit{Note that you must assign a page to each problem you submit.}}}   Gradescope has great YouTube videos available on how to submit homework.  \textit{\textbf{Failure to submit homework correctly will result in a zero on homework.}}
\item Problems that appear with the notation \colorbox{yellow}{$\ast$} will require you to TeX your solution.  If no highlighted star appears, then a hand written solution is OK.  
\item Late homework is not accepted.  Lateness due to technical issues will not be excused.  
\end{enumerate}

\hrule

\vspace{0.5cm}



\begin{enumerate}
    \item Find the projection matrices $P$ onto the line passing through the origin in the direction of $\vec v = (-1, 2, 2)$.  Verify that $P(I - P) = 0$.  Does this make sense?   
    \\
    \textbf{Solution:}\\
    We can express the line as follows
    \[
    L(t)=t(-1,2,2)
    \]
    And to find its projection matrix onto this line, we are only interested in the direction vector
    \[
    \vec v\ = (-1,2,2)
    \]
    let $\vec b$ be an arbitrary vector, to obtain its projection onto $\vec v$, the follow has to stands
    \[
    \vec{P}_b = \vec{v}x
    \]
    where $\vec P$ is the projection matrix, and it takes the following form
    \[
    P=A(A^TA)^{-1}A^T
    \]
    Since we are projecting a vector to a vector, we let $A=\vec v$
    \[
    \therefore \; \vec P\ = 
    \begin{bmatrix}
       -1\\2\\2 
    \end{bmatrix}
    \;\;
    \left(
    \begin{bmatrix}
        -1&2&2
    \end{bmatrix}
    \begin{bmatrix}
       -1\\2\\2 
    \end{bmatrix}
    \right)^{-1}
    \;\;
    \begin{bmatrix}
        -1&2&2
    \end{bmatrix}
    \]
    \[
    \vec P\ = 
    \frac{1}{9}
    \begin{bmatrix}
       -1\\2\\2 
    \end{bmatrix}
    \begin{bmatrix}
        -1&2&2
    \end{bmatrix}
    =
    \frac{1}{9}
    \begin{bmatrix}
        1&-2&-2\\
        -2&4&4\\
        -2&4&4
    \end{bmatrix}
    \]
    now we verify if $P(I-P)=0$
    \[
    I-P=\frac{1}{9}
    \left(
    \begin{bmatrix}
        9&0&0\\
        0&9&0\\
        0&0&9
    \end{bmatrix}
    -
    \begin{bmatrix}
        1&-2&-2\\
        -2&4&4\\
        -2&4&4
    \end{bmatrix}
    \right)
    \]
    \[
    I-P=\frac{1}{9}
    \begin{bmatrix}
        8&2&2\\
        2&5&-4\\
        2&-4&5
    \end{bmatrix}
    \]
    \[
    \therefore P(I-P)=\frac{1}{81}
    \begin{bmatrix}
        1&-2&-2\\
        -2&4&4\\
        -2&4&4
    \end{bmatrix}
    \begin{bmatrix}
        8&2&2\\
        2&5&-4\\
        2&-4&5
    \end{bmatrix}
    =
    \begin{bmatrix}
        0&0&0\\0&0&0\\0&0&0
    \end{bmatrix}
    \]
    which is indeed $\vec 0$, and this result make sense.\\
    As $P$ projects the vector onto vector $\vec v$, $I-P$ projects a vector to a subspace orthogonal to that of $P$, and since $P \perp (I-P)$, projecting $P$ onto $I-P$ will yield a $\vec 0$
\\ \\ \\ 




    
	\item Let $S$ be the span of the vectors $\vec v = (2,2,-1)$, and $\vec w = (2,-1,2)$.  Let $S^{\perp}$ be the space that is orthogonal to $S$.  That is, every vector in $S$ is perpendicular to any vector in $S^{\perp}$.  Find the projection matrix onto $S^{\perp}$.  \\
	\textbf{Solution:}\\
             First, we would find the span of the space composed of basis $\vec v\ ,\vec w$
             \[
             \text{let } A =
             \begin{bmatrix}
                 2&2\\
                 2&-1\\
                 -1&2
             \end{bmatrix}
             \]
             The span of $\{\vec v\ ,\vec w\ \}$ is the column space $C(A)$, and we can see that $\vec w$ is not a linear combination of $\vec v$
             \[
             \therefore \; Span(C(A))=S
             \]
             We know that the subspace that is orthogonal to $C(A)$ is $N(A^T)$, so we now obtain the nullspace of $A^T$
             \[
             A^T=
             \begin{bmatrix}
                 2&2&-1\\
                 2&-1&2
             \end{bmatrix}
             \]
             \[
             \text{let }A^Tx=0
             \]
             \[
             \begin{bmatrix}
                 2&2&-1\\
                 2&-1&2
             \end{bmatrix}
             x=0
             \]
             the matrix can be reduced to the following
             \[
             \begin{bmatrix}
                 2&2&-1\\
                 0&3&-3
             \end{bmatrix}
             \]
             \[
             \text{let } x_3=t, \text{ we can see that }x_2=t
             \]
             \[
             \text{and } x_1=-\frac{1}{2}t
             \]
             \[
             \therefore \; 
             2\vec x\ = 
             t\begin{bmatrix}
                 -1\\2\\2
             \end{bmatrix}
             \;\;\; \text{(scaled by a factor of 2)}
             \]
             \[
             \therefore \; S^\perp = N(A^T) = 
             \begin{bmatrix}
                 -1\\2\\2
             \end{bmatrix}
             \]
            and we have already found the projection matrix for this subspace in \textit{problem 1}
            \[
            \therefore \; \vec P\ = 
                \frac{1}{9}
                \begin{bmatrix}
                    1&-2&-2\\
                    -2&4&4\\
                    -2&4&4
                \end{bmatrix}
            \]
            \\ \\ \\ 
	
	\item  Find the projection of $\vec b=(4,3,1,0)$ onto the nullspace of the matrix 
	\[A=\begin{bmatrix} 1&1&1&1\\-2&-1&0&2\end{bmatrix} .\]
        \textbf{Solution:}\\
        First, we would find the nullspace of the matrix with the following
        \[
        Ax=0
        \]
        \[
        \begin{bmatrix} 1&1&1&1\\-2&-1&0&2\end{bmatrix}x=0
        \]
        we find the REF form of the matrix $A$
        \[
        \begin{bmatrix}
            1&1&1&1\\
            0&1&2&3
        \end{bmatrix}
        \]
        \[
        \text{let }x_4 = s \;\;\; x_3=t
        \]
        \[
        x_2=-2t-3s
        \]
        \[
        x_1 -2t-3s-t-s=0
        \]
        \[
        x_1=3t+4s
        \]
        \[
        x=
        t
        \begin{bmatrix}
            3\\-2\\1\\0
        \end{bmatrix}
        +s
        \begin{bmatrix}
            4\\-3\\0\\1
        \end{bmatrix}
        \]
        \[
        \therefore \; N(A)= span \left(
        \begin{bmatrix}
            3\\-2\\1\\0
        \end{bmatrix}
        \;,
        \begin{bmatrix}
            4\\-3\\0\\1
        \end{bmatrix}
        \right)
        \]
        \[
        \text{let }B=
        \begin{bmatrix}
            3&4\\
            -2&-3\\
            1&0\\
            0&1
        \end{bmatrix}
        \]
        Now we go through the same procedure as before to find the projection matrix
        \[
        P = B(B^TB)^{-1}B^T
        \]
        \[
        (B^TB)^{-1} = \left( 
        \begin{bmatrix}
            3&4\\
            -2&-3\\
            1&0\\
            0&1
        \end{bmatrix}^T
        \begin{bmatrix}
            3&4\\
            -2&-3\\
            1&0\\
            0&1
        \end{bmatrix}
        \right)^{-1}
        \]
        \[
        (B^TB)^{-1}= 
        \begin{bmatrix}
            14&26\\
            18&26
        \end{bmatrix}^{-1}
        =\frac{1}{20}
        \begin{bmatrix}
            13&-9\\
            -9&7
        \end{bmatrix}
        \]
        \[
        B(B^TB)^{-1}B^T= \frac{1}{20}
        \begin{bmatrix}
            3&4\\
            -2&-3\\
            1&0\\
            0&1
        \end{bmatrix}
        \begin{bmatrix}
            13&-9\\
            -9&7
        \end{bmatrix}
        \begin{bmatrix}
            3&4\\
            -2&-3\\
            1&0\\
            0&1
        \end{bmatrix}^T
        \]
        \[
        =
        \frac{1}{20}
        \begin{bmatrix}
        13 & -9 & 3 & 1 \\
        -9 & 7 & 1 & -3 \\
        3 & 1 & 13 & -9 \\
        1 & -3 & -9 & 7
        \end{bmatrix}
        \]
        now we can evaluate $\vec{P}_b$
        \[
        \vec{P}_b = 
        \frac{1}{20}
        \begin{bmatrix}
        13 & -9 & 3 & 1 \\
        -9 & 7 & 1 & -3 \\
        3 & 1 & 13 & -9 \\
        1 & -3 & -9 & 7
        \end{bmatrix}
        \begin{bmatrix}
            4\\3\\1\\0
        \end{bmatrix}
        =
        \frac{1}{10}
        \begin{bmatrix}
            14\\-7\\14\\-7
        \end{bmatrix}
        \]
	\\ \\ \\ 

	\item  Consider the data points $(-2, 3)$, $(2, 1)$, $(3, -4)$, and $(5, 2)$.  Find the equation of the line $y=mx + b$ that best fits these points. \\
    \textbf{Solution:}\\
    a straight line through these points would be in the form as follow
    \[
    y=mx+b
    \]
    we can create a system of equation as below
    \[
        \begin{cases}
            -2x+b=3\\
            2x+b=1\\
            3x+b=-4\\
            5x+b=2
        \end{cases}
    \]
this can be expressed as a coefficient matrix
    \[
    \begin{bmatrix}
        -2&1\\
        2&1\\
        3&1\\
        5&1\\
    \end{bmatrix}
    \begin{bmatrix}
        \hat{m}\\
        \hat{b}
    \end{bmatrix}
    =
    \begin{bmatrix}
        3\\1\\-4\\2
    \end{bmatrix}
    \]
    we can see that these points are not on the same line, therefore the system has no solution\\
    to find the line of best fit, we use the projection matrix to find the closest approximation
    \[
    \text{let } A = 
    \begin{bmatrix}
        -2&1\\
        2&1\\
        3&1\\
        5&1\\
    \end{bmatrix}
    \;\;\; b = 
    \begin{bmatrix}
        3\\1\\-4\\2
    \end{bmatrix}
    \]
    \[
    A\hat{x}=P_b=A(A^TA)^{-1}A^Tb
    \]
    \[
    \hat{x} = (A^TA)^{-1}A^Tb
    \]
    \[
    \hat{x} = \left( 
    \begin{bmatrix}
        -2&1\\
        2&1\\
        3&1\\
        5&1\\
    \end{bmatrix}^T
    \begin{bmatrix}
        -2&1\\
        2&1\\
        3&1\\
        5&1\\
    \end{bmatrix}
    \right)^{-1}
    \begin{bmatrix}
        -2&1\\
        2&1\\
        3&1\\
        5&1\\
    \end{bmatrix}^T
    \begin{bmatrix}
        3\\1\\-4\\2
    \end{bmatrix}
    \]
    \[
    =
    \begin{bmatrix}
    \frac{1}{26} & \frac{-1}{13} \\
    \frac{-1}{13} & \frac{21}{52}
    \end{bmatrix}
    \begin{bmatrix}
    -2 & 2 & 3 & 5 \\
    1 & 1 & 1 & 1
    \end{bmatrix}
    \begin{bmatrix}
        3\\1\\-4\\2
    \end{bmatrix}
    \]
    \[
    =
    \begin{bmatrix}
    \frac{-2}{13} & 0 & \frac{1}{26} & \frac{3}{26} \\
    \frac{29}{52} & \frac{1}{4} & \frac{9}{52} & \frac{1}{52}
    \end{bmatrix}
    \begin{bmatrix}
        3\\1\\-4\\2
    \end{bmatrix}
    \]
    \[
    \hat{x}=
    \begin{bmatrix}
    \frac{-5}{13} \\
    \frac{33}{26}
    \end{bmatrix}
    \]
    therefore the line of best fit is
    \[
    y = -\frac{5}{13}x + \frac{33}{26}
    \]
    \\ \\ \\ 
 



 
		
\item Consider the data points $(-2, 3)$, $(2, 1)$, $(3, -4)$, and $(5, 2)$.
Find the equation of the parabola $y=ax^{2}+bx + c$ that best fits these points.  You may use any software you wish to help you with the matrix multiplication.  
\\
\textbf{Solution:}\\
Given that all parabolas can be written in such form
\[
y = ax^2+bx+c
\]
we can approach this problem like above and set up the following system of equations
\[
\begin{cases}
    4a-2b+c=3\\
    4a+2b+c=1\\
    9a+3b+c=-4\\
    25a+5b+c=2
\end{cases}
\]
we can rewrite the system into a coefficient matrix
\[
\begin{bmatrix}
    4&-2&1\\
    4&2&1\\
    9&3&1\\
    25&5&1
\end{bmatrix}
\begin{bmatrix}
    \hat{a}\\ \hat{b}\\ \hat{c}
\end{bmatrix}
=
\begin{bmatrix}
    3\\1\\-4\\2
\end{bmatrix}
\]
\[
\text{let }A = 
\begin{bmatrix}
    4&-2&1\\
    4&2&1\\
    9&3&1\\
    25&5&1
\end{bmatrix}
\;\;\;\;
b = 
\begin{bmatrix}
    3\\1\\-4\\2
\end{bmatrix}
\]
similar to above, we can obtain the coefficient of the best parabola as below
\[
\hat{x} = (A^TA)^{-1}A^Tb
\]
\[
\hat{x} = 
\left(
\begin{bmatrix}
    4&-2&1\\
    4&2&1\\
    9&3&1\\
    25&5&1
\end{bmatrix}^T
\begin{bmatrix}
    4&-2&1\\
    4&2&1\\
    9&3&1\\
    25&5&1
\end{bmatrix}
\right)^{-1}
\begin{bmatrix}
    4&-2&1\\
    4&2&1\\
    9&3&1\\
    25&5&1
\end{bmatrix}^T
\begin{bmatrix}
    3\\1\\-4\\2
\end{bmatrix}
\]
using matrix calculator, we can obtain $\hat{x}$ to be
\[
\hat{x} = \frac{1}{1549}
\begin{bmatrix}
457 \\
-1791 \\
-442
\end{bmatrix}
\]
therefore, the best parabola can be expressed as followed
\[
y = \frac{1}{1549} \left( 457x^2-1791x-442   \right)
\]
\\



\item Find the best approximation to $\vec{u}=(3,-7,2,3)$ as a linear combination of $\vec v_1=(2,-1,-3,1)$ and $\vec v_2=(1,1,0,-1)$.
\\
\textbf{Solution:}
\\
\[
\text{let }A = [\vec{v}_1 \; \vec{v}_2] = 
\begin{bmatrix}
    2&1\\
    -1&1\\
    -3&0\\
    1&-1
\end{bmatrix}
\]
we want the following relationship to hold
\[
A \Vec{x}=\Vec{u}
\]
we can utilize the relationship in the previous question to obtain $\vec x$ such that it bext approximate $\vec u$
\[
\vec x\ = (A^TA)^{-1}A^T\Vec{u}
\]
\[
(A^TA)^{-1} = 
\left(
\begin{bmatrix}
    2&1\\
    -1&1\\
    -3&0\\
    1&-1
\end{bmatrix}^T
\begin{bmatrix}
    2&1\\
    -1&1\\
    -3&0\\
    1&-1
\end{bmatrix}
\right)^{-1}
=
\begin{bmatrix}
    14&1\\
    1&3
\end{bmatrix}^{-1}
=
\frac{1}{41}
\begin{bmatrix}
    3&-1\\
    -1&14
\end{bmatrix}
\]
\[
\hat x\ = (A^TA)^{-1}A^T\vec u\ = 
\frac{1}{41}
\begin{bmatrix}
    3&-1\\
    -1&14
\end{bmatrix}
\begin{bmatrix}
    2&1\\
    -1&1\\
    -3&0\\
    1&-1
\end{bmatrix}^T
\begin{bmatrix}
    3\\-7\\2\\3
\end{bmatrix}
=
\frac{1}{41}
\begin{bmatrix}
    28\\-105
\end{bmatrix}
\]
therefore, the best approximation expressed as a linear combination would be
\[
A\vec x\ = \vec{u}
\]
where
\[
A = \begin{bmatrix}
    2&1\\
    -1&1\\
    -3&0\\
    1&-1
\end{bmatrix}
\;\;\;
\vec x\ = 
\frac{1}{41}
\begin{bmatrix}
    28\\-105
\end{bmatrix}
\;\;\;
\vec u\ = 
\begin{bmatrix}
    3\\-7\\2\\3
\end{bmatrix}
\]
\\ \\ \\ \\ \\ \\ \\ 



 

	\item  Consider the matrix 
	\[ A = \begin{bmatrix} 1&-6\\3&6\\4&8\\5&0\\7&8\end{bmatrix}. \]
	\begin{enumerate} 
		\item  Let $\beta = \{\vec q_1, \, \vec q_2, \, \vec q_3, \, \vec q_4, \, \vec q_5\}$ be an orthonormal basis for $\mathbb{R}^5$, such that $\vec q_1, \, \vec q_2$ span the column space of $A$. Which of the four fundamental subspaces is spanned by $\vec q_3, \, \vec q_4, \, \vec q_5$?

        \textbf{Answer: }The subspace that is spanned by $\vec q_3, \, \vec q_4, \, \vec q_5$ is $N(A^T)$
\newline
    
		\item  Find $\vec q_1$ and $\vec q_2$.
        \textbf{Solution:}\\
        To obtain $\vec q_1$ and $\vec q_2$, we would first obtain $\vec w_1$ and $\vec w_2$ with the Gram-Schmidt Process
        \[
        \text{let } 
        A = [\Vec{v}_1 \; \Vec{v}_2]
        \;\;\;
        \vec{w}_1 = \Vec{v}_1 =  
        \begin{bmatrix}
            1\\3\\4\\5\\7
        \end{bmatrix}
        \;\;\; \Vec{w}_2 = 
        \Vec{v}_2 - \left( 
        \frac{w_1^Tv_2}{w_1^Tw_1}
        \right)w_1
        \]
        let us evaluate $w_2$
        \[
        w_2 = 
        \begin{bmatrix}
            -6\\6\\8\\0\\8
        \end{bmatrix}
        -
        \frac{100}{100}
        \begin{bmatrix}
            1\\3\\4\\5\\7
        \end{bmatrix}
        =
        \begin{bmatrix}
            -7\\3\\4\\-5\\1
        \end{bmatrix}
        \]
        now we normalize both $w_1$ and $w_2$ to obtain $q_1$ and $q_2$
        \[
        q_1 = \frac{w_1}{\sqrt{w_1^Tw_1}} = 
        \frac{1}{10} 
        \begin{bmatrix}
            1\\3\\4\\5\\7
        \end{bmatrix}
        \]
        \[
        q_2 = \frac{w_2}{\sqrt{w_2^Tw_2}} = 
        \frac{1}{10}
        \begin{bmatrix}
            -7\\3\\4\\-5\\1
        \end{bmatrix}
        \]
        


        
  
		\item  Find $\beta$. You should find $\vec q_3$, $\vec q_4$, and $\vec q_5$. \\
        \textbf{Solutions:}\\
        First, we identify $N(A^T)$
        \[
        A^T=
        \begin{bmatrix}
            1&3&4&5&7\\
            -6&6&8&0&8
        \end{bmatrix}
        \]
        To find the null space of this matrix, we set up the following
        \[
        Ax=0
        \]
        we first reduce the following matrix
        \[
        \begin{bmatrix}
            1&3&4&5&7\\
            -6&6&8&0&8
        \end{bmatrix}
        \]
        the REF form is
        \[
        \begin{bmatrix}
            1 & 3 & 4 & 5 & 7 \\
            0 & 24 & 32 & 30 & 50
        \end{bmatrix}
        \]
        \[
        \text{let }x_3=u \;\;\; x_4=v \;\;\; x_5=w
        \]
        plugging in, we can obtain
        \[
        x_1=-\frac{5}{4}v+\frac{25}{4}w
        \]
        \[
        x_2=-\frac{4}{3}u-\frac{5}{3}v-\frac{25}{12}w
        \]
        from these, we can extrapolate the basis for null space to be
        \[
        N(A^T): \left(
        \begin{bmatrix}
            0\\ -4\\3\\0\\0
        \end{bmatrix}
        \;,\;
        \begin{bmatrix}
            -5\\ -5\\ 0\\4\\0
        \end{bmatrix}
        \;,\;
        \begin{bmatrix}
            -9\\ -25\\0\\0\\12
        \end{bmatrix}
        \right)
        \]
        we let $v_3$, $v_4$, and $v_5$ denote the 3 vectors that formed the basis of the null space, and let $w_3=v_3$
        \\
        we would use the Gram-Schmidt process to find $w_4$ and $w_5$ respectively
        \[
        w_4=v_4- \left( \frac{w_3^Tv_4}{w_3^Tw_3} \right)w_3
        \]
        \[
        w_4=
        \begin{bmatrix}
            -5\\ -5\\ 0\\4\\0
        \end{bmatrix}
        -
        \frac{4}{5}
        \begin{bmatrix}
            0\\-4\\3\\0\\0
        \end{bmatrix}
        =
        \frac{1}{5}
        \begin{bmatrix}
           -25\\-9\\-12\\20 \\0
        \end{bmatrix}
        \;\;\;
        \]
        we scale $w_4$ by $5$ such that its elements are integers, and use it to find $w_5$
        \[
        w_5 = v_5- \left( \frac{w_4^Tv_5}{w_4^Tw_4}\right)w_4 - \left( \frac{w_3^Tv_5}{w_3^Tw_3}\right)w_3
        \]
        \[
        w_5= v_5
        +\frac{2}{3}w_4
        -4w_3
        \]
        \[
        w_5=
        \begin{bmatrix}
        81\\225\\0\\0\\-108   
        \end{bmatrix}\;\;\;, \text{scaled by a factor of 25}
        \]
        now we normalize $w_3,\;w_4,\;w_5$ to obtain $q_1,\;q_2,\;q_3$
        \[
        q_3=
        \frac{1}{5}
        \begin{bmatrix}
            0\\ -4\\ 3 \\0 \\0
        \end{bmatrix}\;\;
        q_4=\frac{1}{\sqrt{1250}}
        \begin{bmatrix}
           -25\\-9\\-12\\20 \\0
        \end{bmatrix}
        \;\;
        q_5=\sqrt{\frac{25}{2754}}
        \begin{bmatrix}
        81\\225\\0\\0\\-108   
        \end{bmatrix}
        \]
        and thus we can determine the matrix $\beta$ is as follows
        \[
        \beta = [q_1,q_2,q_3,q_4,q_5]
        \]
        where
        \[
        q_1=
        \frac{1}{10} 
        \begin{bmatrix}
            1\\3\\4\\5\\7
        \end{bmatrix}\;\;
        q_2=
        \frac{1}{10}
        \begin{bmatrix}
            -7\\3\\4\\-5\\1
        \end{bmatrix}
        \]
        \[
         q_3=
        \frac{1}{5}
        \begin{bmatrix}
            0\\ -4\\ 3 \\0 \\0
        \end{bmatrix}\;\;
        q_4=\frac{1}{\sqrt{1250}}
        \begin{bmatrix}
           -25\\-9\\-12\\20 \\0
        \end{bmatrix}
        \;\;
        q_5=\sqrt{\frac{25}{2754}}
        \begin{bmatrix}
        81\\225\\0\\0\\-108   
        \end{bmatrix}
        \]

  
		\item Find the $Q$ and $R$ in the $QR$-decomposition of $A$.\\
        \textbf{Solution:}\\
        The QR-Decomposition of the matrix takes the following form
        \[
        A=QR
        \]
        where Q is an orthogonal matrix and R is an invertible upper triangular square matrix
        \\
        and we have already obtained the components for Q, and we have Q as shown below
        \[
        Q = [q_1 \;,\;q_2] = 
        \frac{1}{10}
        \begin{bmatrix}
            1&-7\\
            3&3\\
            3&4\\
            5&-5\\
            7&1
        \end{bmatrix}
        \]
        we can obtain $R$ by
        \[
        R=Q^TA
        \]
        \[
        R=
        \begin{bmatrix}
        \frac{1}{10} & \frac{3}{10} & \frac{2}{5} & \frac{1}{2} & \frac{7}{10} \\
        \frac{-7}{10} & \frac{3}{10} & \frac{2}{5} & \frac{-1}{2} & \frac{1}{10}
        \end{bmatrix}
        \begin{bmatrix}
        1 & -6 \\
        3 & 6 \\
        4 & 8 \\
        5 & 0 \\
        7 & 8
        \end{bmatrix}
        =
        \begin{bmatrix}
            10&10\\
            0&10
        \end{bmatrix}
        \]
        And the QR-decomposition of A is
        \[
        A = 
        \frac{1}{10}
        \begin{bmatrix}
            1&-7\\
            3&3\\
            3&4\\
            5&-5\\
            7&1
        \end{bmatrix}
        \begin{bmatrix}
            10&10\\
            0&10
        \end{bmatrix}
        \]


  
		\item  Using the $QR$ decomposition of $A$ to find the least squares solution to $A\vec x = \vec b$, if $\vec b = (-3,7,1,0,4)$.\\
        \textbf{Solution:}\\
        The best solution can be obtained as follows
        \[
        \hat{x} = R^{-1}Q^Tb
        \]
        \[
        R^{-1} = 
        \begin{bmatrix}
        \frac{1}{10} & \frac{-1}{10} \\
        0 & \frac{1}{10}
        \end{bmatrix}
        \]
        \[
        Q^T=
        \begin{bmatrix}
        \frac{1}{10} & \frac{3}{10} & \frac{2}{5} & \frac{1}{2} & \frac{7}{10} \\
        \frac{-7}{10} & \frac{3}{10} & \frac{2}{5} & \frac{-1}{2} & \frac{1}{10}
        \end{bmatrix}
        \]
        \[
        \hat{x} = \begin{bmatrix}
            0\\ \frac{1}{2}
        \end{bmatrix}
        \]
		\end{enumerate} 

        

 \item Let $A = \begin{bmatrix} a&b\\c&d\end{bmatrix}$.  Let $\lambda_{1}, \; \lambda_{2}$ be the eigenvalues of $A$.  
\begin{enumerate}
    \item \colorbox{yellow}{$\ast$}  Show that $\lambda_{1} + \lambda_{2} = a + d$.\\
    \begin{proof} \text{$\lambda_{1} + \lambda_{2} = a + d$}
    \[
    |A-I\lambda|=0
    \]
    \[
    (a-\lambda)(d-\lambda)-bc=0
    \]
    \[
    ad-a\lambda-d\lambda+\lambda^2-bc=0
    \]
    \[
    \lambda^2 -(a+d)\lambda +(ad-bc)=0
    \]
    using the quadratic formula, we can obtain the solutions for $\lambda$
    \[
    \lambda_1= \frac{(a+d)+\sqrt{(a+d)^2-4(ad-bc)}}{2}
    \]
    \[
    \lambda_2= \frac{(a+d)-\sqrt{(a+d)^2-4(ad-bc)}}{2}
    \]
    \[
    \lambda_1+\lambda_2 = \frac{2(a+d)+\sqrt{(a+d)^2-4(ad-bc)}-\sqrt{(a+d)^2-4(ad-bc)}}{2}
    \]
    \[
    \therefore \lambda_1+\lambda_2 = a+d = tr(A)
    \]
    \end{proof}

    
    \item \colorbox{yellow}{$\ast$}  Show that $\lambda_{1} \lambda_{2} = |A|$.
    \begin{proof} $\lambda_{1} \lambda_{2} = |A|$
    
    \[
    \lambda_1\lambda_2 = \frac{ \left[ (a+d)+\sqrt{(a+d)^2-4(ad-bc)} \right] \left[ (a+d)-\sqrt{(a+d)^2-4(ad-bc)} \right] }{4}
    \]
    \[
    \lambda_1\lambda_2 = \frac{(a+d)^2-[(a+d)^2-4(ad-bc)]}{4}
    \]
    \[
    \lambda_1\lambda_2 = \frac{4|A|}{4} = |A|
    \]
    \end{proof}

    
\end{enumerate}

\begin{remark} 
The sum of the diagonal terms of a matrix is called the trace of the matrix and is denoted by $tr(A)$.  In the general setting, the sum of the eigenvalues is equal to the trace of the matrix, while the product of eigenvalues is equal to the determinant. 
\end{remark}

\item Let $A$ be $4 \times 4$ matrix such that 
\[
A \begin{bmatrix} x_{1} \\ x_{2} \\ x_{3} \\ x_{4} \end{bmatrix} = \begin{bmatrix} x_{4} \\ x_{1} \\ x_{2} \\ x_{3} \end{bmatrix} 
\]
Find the eigenvalues of $A$.  \\
\textbf{Solutions:}\\
First, we would identify the matrix A
\[
A = 
\begin{bmatrix}
    0&0&0&1\\
    1&0&0&0\\
    0&1&0&0\\
    0&0&1&0
\end{bmatrix}
\]
now we solve for its eigenvalue
\[
\left|\begin{matrix}
0-\lambda & 0 & 0 & 1 \\
1 & 0-\lambda & 0 & 0 \\
0 & 1 & 0-\lambda & 0 \\
0 & 0 & 1 & 0-\lambda
\end{matrix}\right|
=
\lambda^4-1=0
\]
we notice that $\lambda$ is the 4 roots of unity of a unit circle in the complex plain, therefore
\[
\lambda_1 = 1 \;\;\; \lambda_2 = -1 \;\;\; \lambda_3 = i \;\;\; \lambda_4 = -i
\]
\\ 

\item \colorbox{yellow}{$\ast$} Let $\lambda$ be an eigenvalue of $A$ with corresponding eigenvector $\vec{x}$.  Show that $\lambda^{k}$ is an eigenvalue of $A^{k}$ with corresponding eigenvector $\vec{x}$.  
\\
\textbf{Solution:}\\
Given that $\lambda$ is an eigenvalue of $A$ with corresponding eigenvector $\vec x$
\[
A\vec x\ = \lambda \vec x
\]
multiply both sides by $A$
\[
A(A\Vec{x})=A(\lambda\vec x\ )
\]
since the eigenvector multiplied by a constant is still an eigenvector, we can rewrite the relationship as
\[
A^2\vec x\ = \lambda^2 \vec x
\]
This pattern repeats it self and eventually we would get 
\[
A^k \vec x \ =\lambda^k \vec{x} \;\;\; k\in +\mathbb{Z}
\]
therefore we may conclude that $\lambda^k$ is the eigenvalue for $A^k$





\item  Let $A = \begin{bmatrix} 2&2\\1&3\end{bmatrix}$.  Find the matrices $M, D$ and $D^{-1}$ such that $A = M D M^{-1}$.  Use the diagonalization of $A$ to compute $A^{100}$.\\
\textbf{Solution:}\\
To obtain the matrix $M$ for diagonalization, we would first obtain the eigen values of $A$
\[
\left|
\begin{matrix}
    2-\lambda&2\\
    1&3-\lambda
\end{matrix}
\right|=0
\]
\[
(2-\lambda)(3-\lambda)-2=0
\]
\[
\lambda^2-5\lambda+4=0
\]
\[
(\lambda-1)(\lambda-4)=0
\]
\[
\therefore \; \lambda_1=1 \;\;\; \lambda_2 = 4
\]
now we plug in the eigenvalues into the original matrix
\[
\lambda_1 \rightarrow 
\begin{bmatrix}
    1&2\\
    1&2
\end{bmatrix}
\]
The REF form of the matrix is
\[
\begin{bmatrix}
    1&2\\
    0&0
\end{bmatrix}
\]
and the corresponding eigenvector is
\[
\Vec{v}_1=
\begin{bmatrix}
    -2\\1
\end{bmatrix}
\]
similarly for $\lambda_2$, we obtain
\[
\lambda_2 \rightarrow 
\begin{bmatrix}
    -1&1\\
    0&0
\end{bmatrix}
\]
\[
\therefore \; \Vec{v}_2=
\begin{bmatrix}
    1\\1
\end{bmatrix}
\]
now we can construct the matrix $M$ and $D$ as below
\[
M=
\begin{bmatrix}
    1&-2\\
    1&1
\end{bmatrix}\;\;\;
D=
\begin{bmatrix}
    4&0\\
    0&1
\end{bmatrix}
\]
We now look for the inverse of $M$
\[
M^{-1}=\frac{1}{|M|}
\begin{bmatrix}
    1&2\\
    -1&1
\end{bmatrix}=\frac{1}{3}
\begin{bmatrix}
    1&2\\
    -1&1
\end{bmatrix}
\]
Therefore the diagonalization of $A$ can be written as
\[
A=MDM^{-1}=
\frac{1}{3}
\begin{bmatrix}
    1&-2\\
    1&1
\end{bmatrix}
\begin{bmatrix}
    4&0\\
    0&1
\end{bmatrix}
\begin{bmatrix}
    1&2\\
    -1&1
\end{bmatrix}
\]
And $A^{100}$ can be expressed as follows
\[
A^{100}=MD^{100}M^{-1}=
\frac{1}{3}
\begin{bmatrix}
    1&-2\\
    1&1
\end{bmatrix}
\begin{bmatrix}
    4^{100}&0\\
    0&1
\end{bmatrix}
\begin{bmatrix}
    1&2\\
    -1&1
\end{bmatrix}
\]




\item $A = \begin{bmatrix} 1&2\\2&1\end{bmatrix}$.  Find the matrices $M, D$ and $D^{-1}$ such that $A = M D M^{-1}$.  Use the diagonalization of $A$ to compute $e^{A}$.\\
\textbf{Solution:}\\
Similar to the previous question, we start by finding the eigenvalues
\[
\left|\begin{matrix}
1-\lambda & 2 \\
2 & 1-\lambda
\end{matrix}\right|
=0
\]
\[
(1-\lambda)^2=4
\]
\[
(1-\lambda)=\pm 2
\]
\[
\lambda_1 = -1\;\;\;\lambda_2=3
\]
plugging into the original matrix, we obtain the eigenvectors
\[
\lambda_1 \rightarrow 
\begin{bmatrix}
    2&2\\
    0&0
\end{bmatrix}\;\;\;
\Vec{v}_1 = 
\begin{bmatrix}
    -1\\1
\end{bmatrix}
\]
\[
\lambda_2 \rightarrow 
\begin{bmatrix}
    -2&2\\
    0&0
\end{bmatrix}\;\;\;
\Vec{v}_2 = 
\begin{bmatrix}
    1\\1
\end{bmatrix}
\]
we can therefore construct our $M$ and $D$ matrices
\[
M=
\begin{bmatrix}
-1 & 1 \\
1 & 1
\end{bmatrix}\;\;\;
D=
\begin{bmatrix}
-1 & 0 \\
0 & 3
\end{bmatrix}
\]
now we find the inverse of $M$
\[
M^{-1}=\frac{1}{|M|}
\begin{bmatrix}
1 & -1 \\
-1 & -1
\end{bmatrix}=
\frac{1}{2}
\begin{bmatrix}
-1 & 1 \\
1 & 1
\end{bmatrix}
\]
Therefore the diagonalization can be expressed as follows
\[
A = MDM^{-1} =
\frac{1}{2}
\begin{bmatrix}
-1 & 1 \\
1 & 1
\end{bmatrix}
\begin{bmatrix}
-1 & 0 \\
0 & 3
\end{bmatrix}
\begin{bmatrix}
-1 & 1 \\
1 & 1
\end{bmatrix}
\]
now we investigate $e^A$
\[
e^A=I+A+\frac{A^2}{2!}+\frac{A^3}{3!}\ldots = M(D^{0}+D^{1}+\frac{D^2}{2!}+\frac{D^3}{3!}\ldots)M^{-1}
\]
\[
e^A=
\frac{1}{2}
\begin{bmatrix}
-1 & 1 \\
1 & 1
\end{bmatrix}
\begin{bmatrix}
    \sum_{n=0}^{\infty} \frac{(-1)^n}{n!}&0\\
    0&\sum_{n=0}^{\infty} \frac{(3)^n}{n!}
\end{bmatrix}
\begin{bmatrix}
-1 & 1 \\
1 & 1
\end{bmatrix}
\]
And now we can express $e^A$ as 
\[
e^{A}= Me^DM^{-1}=
\frac{1}{2}
\begin{bmatrix}
-1 & 1 \\
1 & 1
\end{bmatrix}
\begin{bmatrix}
e^{-1} & 0 \\
0 & e^3
\end{bmatrix}
\begin{bmatrix}
-1 & 1 \\
1 & 1
\end{bmatrix}
\]





\item Assume $A$ is an $n \times n$ matrix such that $A^{2} = 5 A$.  Find $e^{A}$.\\
\textbf{Solution}\\
From the given condition we may deduce that 
\[
A^3=5A^2
\]
\[
A^4=5A^3 \;\;\ldots etc. 
\]
Now we expand $e^A$ with its Taylor series
\[
e^A=I+A+\frac{A^2}{2!}+\frac{A^3}{3!}\ldots
\]
we substitute the above relationship we found
\[
e^A=I+A+\frac{5A}{2!}+\frac{5A^2}{3!}+\frac{5A^3}{4!}\ldots 
\]
we notice that 
\[
5A^2=5(5A)=5^2A
\]
\[
5A^3=5A(A^2)=5^2A^2=5^3A
\]
\[
5A^4=5(5A)(5A)=5^4A
\]
now we can regroup the series
\[
e^A=I+A(1+\frac{5}{2!}+\frac{5^2}{3!}+\frac{5^3}{4!}\ldots )
\]
and by rewriting the series into an infinite sum, we obtain
\[
e^A=I+A\left(
\sum_{n=1}^{\infty} \frac{5^{n-1}}{n!}
\right)
\]
we investigate this infinite sum
\[
\sum_{n=1}^{\infty} \frac{5^{n-1}}{n!} = \frac{1}{5}\sum_{n=1}^{\infty} \frac{5^{n}}{n!} = \frac{1}{5}(e^5-1)
\]
\[
\therefore \; e^A = I+\frac{e^5-1}{5}A
\]

	
\end{enumerate}



\end{document}